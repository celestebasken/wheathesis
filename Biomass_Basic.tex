% Options for packages loaded elsewhere
\PassOptionsToPackage{unicode}{hyperref}
\PassOptionsToPackage{hyphens}{url}
\documentclass[
]{article}
\usepackage{xcolor}
\usepackage[margin=1in]{geometry}
\usepackage{amsmath,amssymb}
\setcounter{secnumdepth}{-\maxdimen} % remove section numbering
\usepackage{iftex}
\ifPDFTeX
  \usepackage[T1]{fontenc}
  \usepackage[utf8]{inputenc}
  \usepackage{textcomp} % provide euro and other symbols
\else % if luatex or xetex
  \usepackage{unicode-math} % this also loads fontspec
  \defaultfontfeatures{Scale=MatchLowercase}
  \defaultfontfeatures[\rmfamily]{Ligatures=TeX,Scale=1}
\fi
\usepackage{lmodern}
\ifPDFTeX\else
  % xetex/luatex font selection
\fi
% Use upquote if available, for straight quotes in verbatim environments
\IfFileExists{upquote.sty}{\usepackage{upquote}}{}
\IfFileExists{microtype.sty}{% use microtype if available
  \usepackage[]{microtype}
  \UseMicrotypeSet[protrusion]{basicmath} % disable protrusion for tt fonts
}{}
\makeatletter
\@ifundefined{KOMAClassName}{% if non-KOMA class
  \IfFileExists{parskip.sty}{%
    \usepackage{parskip}
  }{% else
    \setlength{\parindent}{0pt}
    \setlength{\parskip}{6pt plus 2pt minus 1pt}}
}{% if KOMA class
  \KOMAoptions{parskip=half}}
\makeatother
\usepackage{color}
\usepackage{fancyvrb}
\newcommand{\VerbBar}{|}
\newcommand{\VERB}{\Verb[commandchars=\\\{\}]}
\DefineVerbatimEnvironment{Highlighting}{Verbatim}{commandchars=\\\{\}}
% Add ',fontsize=\small' for more characters per line
\usepackage{framed}
\definecolor{shadecolor}{RGB}{248,248,248}
\newenvironment{Shaded}{\begin{snugshade}}{\end{snugshade}}
\newcommand{\AlertTok}[1]{\textcolor[rgb]{0.94,0.16,0.16}{#1}}
\newcommand{\AnnotationTok}[1]{\textcolor[rgb]{0.56,0.35,0.01}{\textbf{\textit{#1}}}}
\newcommand{\AttributeTok}[1]{\textcolor[rgb]{0.13,0.29,0.53}{#1}}
\newcommand{\BaseNTok}[1]{\textcolor[rgb]{0.00,0.00,0.81}{#1}}
\newcommand{\BuiltInTok}[1]{#1}
\newcommand{\CharTok}[1]{\textcolor[rgb]{0.31,0.60,0.02}{#1}}
\newcommand{\CommentTok}[1]{\textcolor[rgb]{0.56,0.35,0.01}{\textit{#1}}}
\newcommand{\CommentVarTok}[1]{\textcolor[rgb]{0.56,0.35,0.01}{\textbf{\textit{#1}}}}
\newcommand{\ConstantTok}[1]{\textcolor[rgb]{0.56,0.35,0.01}{#1}}
\newcommand{\ControlFlowTok}[1]{\textcolor[rgb]{0.13,0.29,0.53}{\textbf{#1}}}
\newcommand{\DataTypeTok}[1]{\textcolor[rgb]{0.13,0.29,0.53}{#1}}
\newcommand{\DecValTok}[1]{\textcolor[rgb]{0.00,0.00,0.81}{#1}}
\newcommand{\DocumentationTok}[1]{\textcolor[rgb]{0.56,0.35,0.01}{\textbf{\textit{#1}}}}
\newcommand{\ErrorTok}[1]{\textcolor[rgb]{0.64,0.00,0.00}{\textbf{#1}}}
\newcommand{\ExtensionTok}[1]{#1}
\newcommand{\FloatTok}[1]{\textcolor[rgb]{0.00,0.00,0.81}{#1}}
\newcommand{\FunctionTok}[1]{\textcolor[rgb]{0.13,0.29,0.53}{\textbf{#1}}}
\newcommand{\ImportTok}[1]{#1}
\newcommand{\InformationTok}[1]{\textcolor[rgb]{0.56,0.35,0.01}{\textbf{\textit{#1}}}}
\newcommand{\KeywordTok}[1]{\textcolor[rgb]{0.13,0.29,0.53}{\textbf{#1}}}
\newcommand{\NormalTok}[1]{#1}
\newcommand{\OperatorTok}[1]{\textcolor[rgb]{0.81,0.36,0.00}{\textbf{#1}}}
\newcommand{\OtherTok}[1]{\textcolor[rgb]{0.56,0.35,0.01}{#1}}
\newcommand{\PreprocessorTok}[1]{\textcolor[rgb]{0.56,0.35,0.01}{\textit{#1}}}
\newcommand{\RegionMarkerTok}[1]{#1}
\newcommand{\SpecialCharTok}[1]{\textcolor[rgb]{0.81,0.36,0.00}{\textbf{#1}}}
\newcommand{\SpecialStringTok}[1]{\textcolor[rgb]{0.31,0.60,0.02}{#1}}
\newcommand{\StringTok}[1]{\textcolor[rgb]{0.31,0.60,0.02}{#1}}
\newcommand{\VariableTok}[1]{\textcolor[rgb]{0.00,0.00,0.00}{#1}}
\newcommand{\VerbatimStringTok}[1]{\textcolor[rgb]{0.31,0.60,0.02}{#1}}
\newcommand{\WarningTok}[1]{\textcolor[rgb]{0.56,0.35,0.01}{\textbf{\textit{#1}}}}
\usepackage{graphicx}
\makeatletter
\newsavebox\pandoc@box
\newcommand*\pandocbounded[1]{% scales image to fit in text height/width
  \sbox\pandoc@box{#1}%
  \Gscale@div\@tempa{\textheight}{\dimexpr\ht\pandoc@box+\dp\pandoc@box\relax}%
  \Gscale@div\@tempb{\linewidth}{\wd\pandoc@box}%
  \ifdim\@tempb\p@<\@tempa\p@\let\@tempa\@tempb\fi% select the smaller of both
  \ifdim\@tempa\p@<\p@\scalebox{\@tempa}{\usebox\pandoc@box}%
  \else\usebox{\pandoc@box}%
  \fi%
}
% Set default figure placement to htbp
\def\fps@figure{htbp}
\makeatother
\setlength{\emergencystretch}{3em} % prevent overfull lines
\providecommand{\tightlist}{%
  \setlength{\itemsep}{0pt}\setlength{\parskip}{0pt}}
\usepackage{bookmark}
\IfFileExists{xurl.sty}{\usepackage{xurl}}{} % add URL line breaks if available
\urlstyle{same}
\hypersetup{
  pdftitle={Biomass\_Basic\_Berries},
  pdfauthor={Celeste Basken},
  hidelinks,
  pdfcreator={LaTeX via pandoc}}

\title{Biomass\_Basic\_Berries}
\author{Celeste Basken}
\date{2026-01-27}

\begin{document}
\maketitle

\begin{verbatim}
## -- Attaching core tidyverse packages ------------------------ tidyverse 2.0.0 --
## v dplyr     1.1.4     v readr     2.1.5
## v forcats   1.0.0     v stringr   1.5.1
## v ggplot2   3.5.2     v tibble    3.2.1
## v lubridate 1.9.4     v tidyr     1.3.1
## v purrr     1.0.4     
## -- Conflicts ------------------------------------------ tidyverse_conflicts() --
## x dplyr::filter() masks stats::filter()
## x dplyr::lag()    masks stats::lag()
## i Use the conflicted package (<http://conflicted.r-lib.org/>) to force all conflicts to become errors
\end{verbatim}

\section{Berries}\label{berries}

\begin{Shaded}
\begin{Highlighting}[]
\CommentTok{\# Loading in data to explore}
\NormalTok{berries }\OtherTok{\textless{}{-}} \FunctionTok{read.csv}\NormalTok{(}\StringTok{"Biomass/Berry\_Weights.csv"}\NormalTok{)}

\NormalTok{berries }\OtherTok{\textless{}{-}}\NormalTok{ berries }\SpecialCharTok{\%\textgreater{}\%}
  \FunctionTok{mutate}\NormalTok{(}
    \AttributeTok{Inoculation =} \FunctionTok{factor}\NormalTok{(Inoculation),}
    \AttributeTok{Polyculture =} \FunctionTok{factor}\NormalTok{(Polyculture),}
    \AttributeTok{Group =} \FunctionTok{interaction}\NormalTok{(Inoculation, Polyculture, }\AttributeTok{sep =} \StringTok{"\_"}\NormalTok{)}
\NormalTok{  )}
\end{Highlighting}
\end{Shaded}

\subsection{Part 1: Pretty basic
figures}\label{part-1-pretty-basic-figures}

Clustering without regard to PPT:

\begin{itemize}
\tightlist
\item
  Just a note here: I'm using Biomass\_g\_pp, which is grams per plant
  of wheat berries. THIS IS IMPORTANT because I forgot to do it at first
  which meant half the treatments were doubled since this should be
  considered per capita. Alas.
\end{itemize}

\pandocbounded{\includegraphics[keepaspectratio]{Biomass_Basic_files/figure-latex/pressure-1.pdf}}

\begin{itemize}
\tightlist
\item
  In the most basic analysis, the monoculture was about half as
  productive
\end{itemize}

The same data is below, but with each pot plotted so you can see the
variation within the treatments:

\pandocbounded{\includegraphics[keepaspectratio]{Biomass_Basic_files/figure-latex/unnamed-chunk-3-1.pdf}}

\subsection{Part 2: Looking at PPT and what the general behavior
is}\label{part-2-looking-at-ppt-and-what-the-general-behavior-is}

To begin, here's what I would EXPECT the graph to look like:

\begin{Shaded}
\begin{Highlighting}[]
\NormalTok{knitr}\SpecialCharTok{::}\FunctionTok{include\_graphics}\NormalTok{(}\StringTok{"Biomass/Berries/expected.png"}\NormalTok{)}
\end{Highlighting}
\end{Shaded}

\pandocbounded{\includegraphics[keepaspectratio]{Biomass/Berries/expected.png}}
Here's the actual figure with the data:

\pandocbounded{\includegraphics[keepaspectratio]{Biomass_Basic_files/figure-latex/unnamed-chunk-5-1.pdf}}
Takeaways

\begin{itemize}
\tightlist
\item
  The Inoc\_inter and strl\_inter groups are behaving as expected, but
  the inoc\_mono and strl\_mono are not (especially strl\_mono, which is
  interesting since it is to an extreme of our treatments)
\item
  This may be explained by intercropping vs monocropping, potentially
\item
  Inoculation seems less significant than intercropping at first pass.
\item
  This kind of makes sense, since that factor is less straightforward.
  Plus, intercropped plants probably had more room, since the faba
  didn't do so well.
\end{itemize}

\subsection{Analyzing these berry biomass trends more in
depth}\label{analyzing-these-berry-biomass-trends-more-in-depth}

All of the treatments together looking at water: both as a bar graph and
boxplot

\begin{Shaded}
\begin{Highlighting}[]
\CommentTok{\# Bar graph looking at water with the individual points plotted on trop}
\FunctionTok{ggplot}\NormalTok{(berries, }\FunctionTok{aes}\NormalTok{(}\AttributeTok{x =}\NormalTok{ Water, }\AttributeTok{y =}\NormalTok{ Biomass\_g\_pp)) }\SpecialCharTok{+}
  \FunctionTok{stat\_summary}\NormalTok{(}\AttributeTok{fun =}\NormalTok{ mean, }\AttributeTok{geom =} \StringTok{"col"}\NormalTok{, }\AttributeTok{width =} \FloatTok{0.6}\NormalTok{) }\SpecialCharTok{+}
  \FunctionTok{geom\_jitter}\NormalTok{(}\AttributeTok{width =} \FloatTok{0.15}\NormalTok{, }\AttributeTok{alpha =} \FloatTok{0.6}\NormalTok{) }\SpecialCharTok{+}
  \FunctionTok{ggtitle}\NormalTok{(}\StringTok{"Berry biomass by PPT, all treatments"}\NormalTok{)}
\end{Highlighting}
\end{Shaded}

\pandocbounded{\includegraphics[keepaspectratio]{Biomass_Basic_files/figure-latex/unnamed-chunk-6-1.pdf}}

\begin{Shaded}
\begin{Highlighting}[]
\CommentTok{\# The same thing (kind of) but as a boxplot with the scattered points}

\CommentTok{\#This code is redundant but good for copy paste later?}
\NormalTok{berries }\OtherTok{\textless{}{-}}\NormalTok{ berries }\SpecialCharTok{\%\textgreater{}\%}
  \FunctionTok{mutate}\NormalTok{(}
    \AttributeTok{Water =} \FunctionTok{factor}\NormalTok{(Water, }\AttributeTok{levels =} \FunctionTok{c}\NormalTok{(}\StringTok{"Control"}\NormalTok{, }\StringTok{"Moderate"}\NormalTok{, }\StringTok{"Severe"}\NormalTok{))}
\NormalTok{  )}

\CommentTok{\# The plot}
\FunctionTok{ggplot}\NormalTok{(berries, }\FunctionTok{aes}\NormalTok{(}\AttributeTok{x =}\NormalTok{ Water, }\AttributeTok{y =}\NormalTok{ Biomass\_g\_pp)) }\SpecialCharTok{+}
  \FunctionTok{geom\_boxplot}\NormalTok{(}\AttributeTok{outlier.shape =} \ConstantTok{NA}\NormalTok{) }\SpecialCharTok{+}
  \FunctionTok{geom\_jitter}\NormalTok{(}\AttributeTok{width =} \FloatTok{0.15}\NormalTok{, }\AttributeTok{alpha =} \FloatTok{0.6}\NormalTok{) }\SpecialCharTok{+}
  \FunctionTok{ggtitle}\NormalTok{(}\StringTok{"Berry biomass by PPT, all treatments"}\NormalTok{)}
\end{Highlighting}
\end{Shaded}

\pandocbounded{\includegraphics[keepaspectratio]{Biomass_Basic_files/figure-latex/unnamed-chunk-7-1.pdf}}

Takeaways

\begin{itemize}
\tightlist
\item
  There is a difference between severe compared to control and moderate,
  but not really within control and moderate
\item
  Can I average the control and moderate somehow? Should I? Or should I
  just remove moderate and have 2 levels of PPT.
\end{itemize}

\subsection{Focusing on crop diversity: inter vs
mono}\label{focusing-on-crop-diversity-inter-vs-mono}

Since the main difference in expectation vs data for the berry biomass
was in the monocropped treatments, I decided to break down these
boxplots by cropping diversity. Here are the individual boxplots, and
then side by side:

These first figures combine inoc and strl. First, just monocropped:

\begin{Shaded}
\begin{Highlighting}[]
\CommentTok{\# Looking at just monocropped}

\NormalTok{berries }\OtherTok{\textless{}{-}}\NormalTok{ berries }\SpecialCharTok{\%\textgreater{}\%}
  \FunctionTok{mutate}\NormalTok{(}
    \AttributeTok{Water =} \FunctionTok{factor}\NormalTok{(Water, }\AttributeTok{levels =} \FunctionTok{c}\NormalTok{(}\StringTok{"Control"}\NormalTok{, }\StringTok{"Moderate"}\NormalTok{, }\StringTok{"Severe"}\NormalTok{)),}
    \AttributeTok{Inoculation =} \FunctionTok{factor}\NormalTok{(Inoculation),}
    \AttributeTok{Polyculture =} \FunctionTok{factor}\NormalTok{(Polyculture)}
\NormalTok{  )}

\NormalTok{berries\_inter }\OtherTok{\textless{}{-}}\NormalTok{ berries }\SpecialCharTok{\%\textgreater{}\%}
  \FunctionTok{filter}\NormalTok{(Polyculture }\SpecialCharTok{==} \StringTok{"Mono"}\NormalTok{)}

\FunctionTok{ggplot}\NormalTok{(berries\_inter, }\FunctionTok{aes}\NormalTok{(}\AttributeTok{x =}\NormalTok{ Water, }\AttributeTok{y =}\NormalTok{ Biomass\_g\_pp)) }\SpecialCharTok{+}
  \FunctionTok{geom\_boxplot}\NormalTok{(}\AttributeTok{outlier.shape =} \ConstantTok{NA}\NormalTok{) }\SpecialCharTok{+}
  \FunctionTok{geom\_jitter}\NormalTok{(}\AttributeTok{width =} \FloatTok{0.15}\NormalTok{, }\AttributeTok{alpha =} \FloatTok{0.6}\NormalTok{) }\SpecialCharTok{+}
  \FunctionTok{ggtitle}\NormalTok{(}\StringTok{"Berry biomass by PPT, Monocropped treatments only"}\NormalTok{)}
\end{Highlighting}
\end{Shaded}

\pandocbounded{\includegraphics[keepaspectratio]{Biomass_Basic_files/figure-latex/unnamed-chunk-8-1.pdf}}

And looking at intercropped

\begin{Shaded}
\begin{Highlighting}[]
\CommentTok{\# Looking at just intercropped}

\NormalTok{berries }\OtherTok{\textless{}{-}}\NormalTok{ berries }\SpecialCharTok{\%\textgreater{}\%}
  \FunctionTok{mutate}\NormalTok{(}
    \AttributeTok{Water =} \FunctionTok{factor}\NormalTok{(Water, }\AttributeTok{levels =} \FunctionTok{c}\NormalTok{(}\StringTok{"Control"}\NormalTok{, }\StringTok{"Moderate"}\NormalTok{, }\StringTok{"Severe"}\NormalTok{)),}
    \AttributeTok{Inoculation =} \FunctionTok{factor}\NormalTok{(Inoculation),}
    \AttributeTok{Polyculture =} \FunctionTok{factor}\NormalTok{(Polyculture)}
\NormalTok{  )}

\NormalTok{berries\_inter }\OtherTok{\textless{}{-}}\NormalTok{ berries }\SpecialCharTok{\%\textgreater{}\%}
  \FunctionTok{filter}\NormalTok{(Polyculture }\SpecialCharTok{==} \StringTok{"Inter"}\NormalTok{)}

\FunctionTok{ggplot}\NormalTok{(berries\_inter, }\FunctionTok{aes}\NormalTok{(}\AttributeTok{x =}\NormalTok{ Water, }\AttributeTok{y =}\NormalTok{ Biomass\_g\_pp)) }\SpecialCharTok{+}
  \FunctionTok{geom\_boxplot}\NormalTok{(}\AttributeTok{outlier.shape =} \ConstantTok{NA}\NormalTok{) }\SpecialCharTok{+}
  \FunctionTok{geom\_jitter}\NormalTok{(}\AttributeTok{width =} \FloatTok{0.15}\NormalTok{, }\AttributeTok{alpha =} \FloatTok{0.6}\NormalTok{) }\SpecialCharTok{+} 
  \FunctionTok{ggtitle}\NormalTok{(}\StringTok{"Berry biomass by PPT, Intercropped treatments only"}\NormalTok{)}
\end{Highlighting}
\end{Shaded}

\pandocbounded{\includegraphics[keepaspectratio]{Biomass_Basic_files/figure-latex/unnamed-chunk-9-1.pdf}}

Let's see that side by side and broken down by inoculation status as
well

\begin{Shaded}
\begin{Highlighting}[]
\CommentTok{\# Splitting it by crop treatment}

\NormalTok{berries }\OtherTok{\textless{}{-}}\NormalTok{ berries }\SpecialCharTok{\%\textgreater{}\%}
  \FunctionTok{mutate}\NormalTok{(}
    \AttributeTok{Water       =} \FunctionTok{factor}\NormalTok{(Water, }\AttributeTok{levels =} \FunctionTok{c}\NormalTok{(}\StringTok{"Control"}\NormalTok{, }\StringTok{"Moderate"}\NormalTok{, }\StringTok{"Severe"}\NormalTok{)),}
    \AttributeTok{Inoculation =} \FunctionTok{factor}\NormalTok{(Inoculation, }\AttributeTok{levels =} \FunctionTok{c}\NormalTok{(}\StringTok{"Strl"}\NormalTok{, }\StringTok{"Inoc"}\NormalTok{)),}
    \AttributeTok{Polyculture =} \FunctionTok{factor}\NormalTok{(Polyculture, }\AttributeTok{levels =} \FunctionTok{c}\NormalTok{(}\StringTok{"Inter"}\NormalTok{, }\StringTok{"Mono"}\NormalTok{))}
\NormalTok{  )}

\FunctionTok{ggplot}\NormalTok{(berries, }\FunctionTok{aes}\NormalTok{(}\AttributeTok{x =}\NormalTok{ Water, }\AttributeTok{y =}\NormalTok{ Biomass\_g\_pp, }\AttributeTok{fill =}\NormalTok{ Inoculation)) }\SpecialCharTok{+}
  \FunctionTok{geom\_boxplot}\NormalTok{(}
    \AttributeTok{outlier.shape =} \ConstantTok{NA}\NormalTok{,}
    \AttributeTok{position =} \FunctionTok{position\_dodge}\NormalTok{(}\AttributeTok{width =} \FloatTok{0.8}\NormalTok{)}
\NormalTok{  ) }\SpecialCharTok{+}
  \FunctionTok{geom\_point}\NormalTok{(}
    \FunctionTok{aes}\NormalTok{(}\AttributeTok{color =}\NormalTok{ Inoculation),}
    \AttributeTok{alpha =} \FloatTok{0.6}\NormalTok{,}
    \AttributeTok{size =} \FloatTok{1.8}\NormalTok{,}
    \AttributeTok{position =} \FunctionTok{position\_dodge}\NormalTok{(}\AttributeTok{width =} \FloatTok{0.8}\NormalTok{)}
\NormalTok{  ) }\SpecialCharTok{+}
  \FunctionTok{facet\_wrap}\NormalTok{(}\SpecialCharTok{\textasciitilde{}}\NormalTok{ Polyculture) }\SpecialCharTok{+}
  \FunctionTok{theme\_bw}\NormalTok{() }\SpecialCharTok{+}
  \FunctionTok{labs}\NormalTok{(}
    \AttributeTok{x =} \StringTok{"Water treatment"}\NormalTok{,}
    \AttributeTok{y =} \StringTok{"Biomass (g)"}
\NormalTok{  ) }\SpecialCharTok{+}
  \FunctionTok{ggtitle}\NormalTok{(}\StringTok{"Both Intercropped and Monoculture Berry biomass by PPT"}\NormalTok{)}
\end{Highlighting}
\end{Shaded}

\pandocbounded{\includegraphics[keepaspectratio]{Biomass_Basic_files/figure-latex/unnamed-chunk-10-1.pdf}}

Takeaways

\begin{itemize}
\tightlist
\item
  The intercropped treatments behave more like I expected (confirming
  first graph)
\item
  In no treatment is there a real difference between control and
  moderate PPT. It may have to be dropped, if that's allowed? Unless
  there is another solution.
\item
  On the bright side, the severe drought worked!
\end{itemize}

\subsection{Berry statistics}\label{berry-statistics}

To conclude my berry work, let's generate some statistics

Before I begin I want to check some assumptions of a t-test!!

The boxes shouldn't have overly long tails or outliers. The residuals
should be straight. These assumptions seem to be met (??).

\begin{itemize}
\tightlist
\item
  I worry that that the residuals look too S-shaped
\end{itemize}

\begin{Shaded}
\begin{Highlighting}[]
\CommentTok{\# Check of distributions}
\FunctionTok{ggplot}\NormalTok{(berries, }\FunctionTok{aes}\NormalTok{(}\AttributeTok{x =}\NormalTok{ Inoculation, }\AttributeTok{y =}\NormalTok{ Biomass\_g\_pp)) }\SpecialCharTok{+}
  \FunctionTok{geom\_boxplot}\NormalTok{() }\SpecialCharTok{+}
  \FunctionTok{geom\_jitter}\NormalTok{(}\AttributeTok{width =} \FloatTok{0.15}\NormalTok{, }\AttributeTok{alpha =} \FloatTok{0.6}\NormalTok{) }\SpecialCharTok{+}
  \FunctionTok{facet\_wrap}\NormalTok{(}\SpecialCharTok{\textasciitilde{}}\NormalTok{ Water) }\SpecialCharTok{+}
  \FunctionTok{theme\_bw}\NormalTok{() }\SpecialCharTok{+}
  \FunctionTok{ggtitle}\NormalTok{(}\StringTok{"Berry biomass by inoculation status"}\NormalTok{)}
\end{Highlighting}
\end{Shaded}

\pandocbounded{\includegraphics[keepaspectratio]{Biomass_Basic_files/figure-latex/unnamed-chunk-11-1.pdf}}

\begin{Shaded}
\begin{Highlighting}[]
\CommentTok{\# Q–Q plots of residuals}
\FunctionTok{ggplot}\NormalTok{(berries, }\FunctionTok{aes}\NormalTok{(}\AttributeTok{sample =}\NormalTok{ Biomass\_g\_pp)) }\SpecialCharTok{+}
  \FunctionTok{stat\_qq}\NormalTok{() }\SpecialCharTok{+}
  \FunctionTok{stat\_qq\_line}\NormalTok{() }\SpecialCharTok{+}
  \FunctionTok{facet\_wrap}\NormalTok{(}\SpecialCharTok{\textasciitilde{}}\NormalTok{ Inoculation }\SpecialCharTok{+}\NormalTok{ Water) }\SpecialCharTok{+}
  \FunctionTok{theme\_bw}\NormalTok{() }\SpecialCharTok{+}
  \FunctionTok{ggtitle}\NormalTok{(}\StringTok{"QQ plot of residuals for inoc status"}\NormalTok{)}
\end{Highlighting}
\end{Shaded}

\pandocbounded{\includegraphics[keepaspectratio]{Biomass_Basic_files/figure-latex/unnamed-chunk-11-2.pdf}}

This next section is generating a variety of summary statistics about
the data, and then saving them as CSVs.

\begin{Shaded}
\begin{Highlighting}[]
\CommentTok{\# Load + clean}
\NormalTok{berries }\OtherTok{\textless{}{-}}\NormalTok{ berries }\SpecialCharTok{\%\textgreater{}\%}
  \FunctionTok{mutate}\NormalTok{(}
    \AttributeTok{Inoculation =} \FunctionTok{factor}\NormalTok{(Inoculation, }\AttributeTok{levels =} \FunctionTok{c}\NormalTok{(}\StringTok{"Strl"}\NormalTok{, }\StringTok{"Inoc"}\NormalTok{)),}
    \AttributeTok{Polyculture =} \FunctionTok{factor}\NormalTok{(Polyculture, }\AttributeTok{levels =} \FunctionTok{c}\NormalTok{(}\StringTok{"Mono"}\NormalTok{, }\StringTok{"Inter"}\NormalTok{)),}
    \AttributeTok{Water =} \FunctionTok{factor}\NormalTok{(Water, }\AttributeTok{levels =} \FunctionTok{c}\NormalTok{(}\StringTok{"Control"}\NormalTok{, }\StringTok{"Moderate"}\NormalTok{, }\StringTok{"Severe"}\NormalTok{))}
\NormalTok{  ) }\SpecialCharTok{\%\textgreater{}\%}
  \FunctionTok{filter}\NormalTok{(}\SpecialCharTok{!}\FunctionTok{is.na}\NormalTok{(Biomass\_g\_pp))}

\CommentTok{\# Descriptive statistics}
\NormalTok{desc\_stats }\OtherTok{\textless{}{-}} \ControlFlowTok{function}\NormalTok{(x) \{}
  \FunctionTok{tibble}\NormalTok{(}
    \AttributeTok{n      =} \FunctionTok{sum}\NormalTok{(}\SpecialCharTok{!}\FunctionTok{is.na}\NormalTok{(x)),}
    \AttributeTok{mean   =} \FunctionTok{mean}\NormalTok{(x, }\AttributeTok{na.rm =} \ConstantTok{TRUE}\NormalTok{),}
    \AttributeTok{sd     =} \FunctionTok{sd}\NormalTok{(x, }\AttributeTok{na.rm =} \ConstantTok{TRUE}\NormalTok{),}
    \AttributeTok{se     =} \FunctionTok{sd}\NormalTok{(x, }\AttributeTok{na.rm =} \ConstantTok{TRUE}\NormalTok{) }\SpecialCharTok{/} \FunctionTok{sqrt}\NormalTok{(}\FunctionTok{sum}\NormalTok{(}\SpecialCharTok{!}\FunctionTok{is.na}\NormalTok{(x))),}
    \AttributeTok{median =} \FunctionTok{median}\NormalTok{(x, }\AttributeTok{na.rm =} \ConstantTok{TRUE}\NormalTok{),}
    \AttributeTok{q25    =} \FunctionTok{quantile}\NormalTok{(x, }\FloatTok{0.25}\NormalTok{, }\AttributeTok{na.rm =} \ConstantTok{TRUE}\NormalTok{, }\AttributeTok{names =} \ConstantTok{FALSE}\NormalTok{),}
    \AttributeTok{q75    =} \FunctionTok{quantile}\NormalTok{(x, }\FloatTok{0.75}\NormalTok{, }\AttributeTok{na.rm =} \ConstantTok{TRUE}\NormalTok{, }\AttributeTok{names =} \ConstantTok{FALSE}\NormalTok{),}
    \AttributeTok{iqr    =} \FunctionTok{IQR}\NormalTok{(x, }\AttributeTok{na.rm =} \ConstantTok{TRUE}\NormalTok{),}
    \AttributeTok{min    =} \FunctionTok{min}\NormalTok{(x, }\AttributeTok{na.rm =} \ConstantTok{TRUE}\NormalTok{),}
    \AttributeTok{max    =} \FunctionTok{max}\NormalTok{(x, }\AttributeTok{na.rm =} \ConstantTok{TRUE}\NormalTok{)}
\NormalTok{  )}
\NormalTok{\}}

\CommentTok{\# Each of the 4 treatments within each Water level}
\NormalTok{summary\_4treat\_by\_water }\OtherTok{\textless{}{-}}\NormalTok{ berries }\SpecialCharTok{\%\textgreater{}\%}
  \FunctionTok{group\_by}\NormalTok{(Water, Inoculation, Polyculture) }\SpecialCharTok{\%\textgreater{}\%}
  \FunctionTok{summarise}\NormalTok{(}\FunctionTok{desc\_stats}\NormalTok{(Biomass\_g\_pp), }\AttributeTok{.groups =} \StringTok{"drop"}\NormalTok{)}

\CommentTok{\# Inoculation only (fold inter/mono together), within each Water}
\NormalTok{summary\_inoc\_by\_water }\OtherTok{\textless{}{-}}\NormalTok{ berries }\SpecialCharTok{\%\textgreater{}\%}
  \FunctionTok{group\_by}\NormalTok{(Water, Inoculation) }\SpecialCharTok{\%\textgreater{}\%}
  \FunctionTok{summarise}\NormalTok{(}\FunctionTok{desc\_stats}\NormalTok{(Biomass\_g\_pp), }\AttributeTok{.groups =} \StringTok{"drop"}\NormalTok{)}

\CommentTok{\# Polyculture only (fold inoc/strl together), within each Water}
\NormalTok{summary\_poly\_by\_water }\OtherTok{\textless{}{-}}\NormalTok{ berries }\SpecialCharTok{\%\textgreater{}\%}
  \FunctionTok{group\_by}\NormalTok{(Water, Polyculture) }\SpecialCharTok{\%\textgreater{}\%}
  \FunctionTok{summarise}\NormalTok{(}\FunctionTok{desc\_stats}\NormalTok{(Biomass\_g\_pp), }\AttributeTok{.groups =} \StringTok{"drop"}\NormalTok{)}

\CommentTok{\# Oerall summaries across Water (all data)}
\NormalTok{summary\_4treat\_overall }\OtherTok{\textless{}{-}}\NormalTok{ berries }\SpecialCharTok{\%\textgreater{}\%}
  \FunctionTok{group\_by}\NormalTok{(Inoculation, Polyculture) }\SpecialCharTok{\%\textgreater{}\%}
  \FunctionTok{summarise}\NormalTok{(}\FunctionTok{desc\_stats}\NormalTok{(Biomass\_g\_pp), }\AttributeTok{.groups =} \StringTok{"drop"}\NormalTok{)}

\NormalTok{summary\_inoc\_overall }\OtherTok{\textless{}{-}}\NormalTok{ berries }\SpecialCharTok{\%\textgreater{}\%}
  \FunctionTok{group\_by}\NormalTok{(Inoculation) }\SpecialCharTok{\%\textgreater{}\%}
  \FunctionTok{summarise}\NormalTok{(}\FunctionTok{desc\_stats}\NormalTok{(Biomass\_g\_pp), }\AttributeTok{.groups =} \StringTok{"drop"}\NormalTok{)}

\NormalTok{summary\_poly\_overall }\OtherTok{\textless{}{-}}\NormalTok{ berries }\SpecialCharTok{\%\textgreater{}\%}
  \FunctionTok{group\_by}\NormalTok{(Polyculture) }\SpecialCharTok{\%\textgreater{}\%}
  \FunctionTok{summarise}\NormalTok{(}\FunctionTok{desc\_stats}\NormalTok{(Biomass\_g\_pp), }\AttributeTok{.groups =} \StringTok{"drop"}\NormalTok{)}


\CommentTok{\# Looking at them as tables}
\NormalTok{summary\_4treat\_by\_water}
\end{Highlighting}
\end{Shaded}

\begin{verbatim}
## # A tibble: 12 x 13
##    Water    Inoculation Polyculture     n  mean     sd     se median   q25   q75
##    <fct>    <fct>       <fct>       <int> <dbl>  <dbl>  <dbl>  <dbl> <dbl> <dbl>
##  1 Control  Strl        Mono            5 0.583 0.225  0.101   0.648 0.596 0.726
##  2 Control  Strl        Inter           5 1.42  0.204  0.0912  1.47  1.24  1.60 
##  3 Control  Inoc        Mono            5 0.880 0.0854 0.0382  0.888 0.840 0.955
##  4 Control  Inoc        Inter           5 1.32  0.544  0.243   1.36  0.895 1.52 
##  5 Moderate Strl        Mono            5 0.779 0.193  0.0863  0.764 0.685 0.894
##  6 Moderate Strl        Inter           5 1.23  0.122  0.0548  1.23  1.16  1.31 
##  7 Moderate Inoc        Mono            5 0.831 0.148  0.0662  0.851 0.832 0.888
##  8 Moderate Inoc        Inter           5 1.40  0.416  0.186   1.40  1.13  1.62 
##  9 Severe   Strl        Mono            5 0.395 0.103  0.0458  0.433 0.418 0.445
## 10 Severe   Strl        Inter           5 0.688 0.251  0.112   0.747 0.742 0.824
## 11 Severe   Inoc        Mono            5 0.141 0.0898 0.0402  0.116 0.105 0.151
## 12 Severe   Inoc        Inter           5 0.683 0.220  0.0984  0.797 0.524 0.837
## # i 3 more variables: iqr <dbl>, min <dbl>, max <dbl>
\end{verbatim}

\begin{Shaded}
\begin{Highlighting}[]
\NormalTok{summary\_inoc\_by\_water}
\end{Highlighting}
\end{Shaded}

\begin{verbatim}
## # A tibble: 6 x 12
##   Water    Inoculation     n  mean    sd     se median   q25   q75   iqr    min
##   <fct>    <fct>       <int> <dbl> <dbl>  <dbl>  <dbl> <dbl> <dbl> <dbl>  <dbl>
## 1 Control  Strl           10 1.00  0.485 0.153   0.958 0.668 1.42  0.747 0.196 
## 2 Control  Inoc           10 1.10  0.435 0.138   0.925 0.852 1.26  0.411 0.738 
## 3 Moderate Strl           10 1.00  0.281 0.0889  1.05  0.796 1.21  0.418 0.526 
## 4 Moderate Inoc           10 1.11  0.419 0.132   0.940 0.859 1.33  0.472 0.590 
## 5 Severe   Strl           10 0.542 0.237 0.0751  0.456 0.421 0.746 0.324 0.215 
## 6 Severe   Inoc           10 0.412 0.327 0.103   0.333 0.125 0.728 0.603 0.0472
## # i 1 more variable: max <dbl>
\end{verbatim}

\begin{Shaded}
\begin{Highlighting}[]
\NormalTok{summary\_poly\_by\_water}
\end{Highlighting}
\end{Shaded}

\begin{verbatim}
## # A tibble: 6 x 12
##   Water    Polyculture     n  mean    sd     se median   q25   q75   iqr    min
##   <fct>    <fct>       <int> <dbl> <dbl>  <dbl>  <dbl> <dbl> <dbl> <dbl>  <dbl>
## 1 Control  Mono           10 0.731 0.224 0.0709  0.752 0.668 0.876 0.208 0.196 
## 2 Control  Inter          10 1.37  0.390 0.123   1.42  1.19  1.58  0.394 0.738 
## 3 Moderate Mono           10 0.805 0.164 0.0520  0.841 0.704 0.893 0.188 0.526 
## 4 Moderate Inter          10 1.31  0.302 0.0955  1.27  1.14  1.39  0.255 0.884 
## 5 Severe   Mono           10 0.268 0.162 0.0512  0.251 0.125 0.429 0.304 0.0472
## 6 Severe   Inter          10 0.685 0.223 0.0704  0.772 0.578 0.834 0.256 0.250 
## # i 1 more variable: max <dbl>
\end{verbatim}

\begin{Shaded}
\begin{Highlighting}[]
\CommentTok{\# Saving these as csvs}

\FunctionTok{write.csv}\NormalTok{(summary\_4treat\_by\_water,}
          \FunctionTok{file.path}\NormalTok{((}\StringTok{"Biomass/Berries"}\NormalTok{), }\StringTok{"summary\_4treat\_by\_water.csv"}\NormalTok{),}
          \AttributeTok{row.names =} \ConstantTok{FALSE}\NormalTok{)}

\FunctionTok{write.csv}\NormalTok{(summary\_inoc\_by\_water,}
          \FunctionTok{file.path}\NormalTok{((}\StringTok{"Biomass/Berries"}\NormalTok{), }\StringTok{"summary\_inoc\_by\_water.csv"}\NormalTok{),}
          \AttributeTok{row.names =} \ConstantTok{FALSE}\NormalTok{)}

\FunctionTok{write.csv}\NormalTok{(summary\_poly\_by\_water,}
          \FunctionTok{file.path}\NormalTok{((}\StringTok{"Biomass/Berries"}\NormalTok{), }\StringTok{"summary\_poly\_by\_water.csv"}\NormalTok{),}
          \AttributeTok{row.names =} \ConstantTok{FALSE}\NormalTok{)}

\FunctionTok{write.csv}\NormalTok{(summary\_4treat\_overall,}
          \FunctionTok{file.path}\NormalTok{((}\StringTok{"Biomass/Berries"}\NormalTok{), }\StringTok{"summary\_4treat\_overall.csv"}\NormalTok{),}
          \AttributeTok{row.names =} \ConstantTok{FALSE}\NormalTok{)}

\FunctionTok{write.csv}\NormalTok{(summary\_inoc\_overall,}
          \FunctionTok{file.path}\NormalTok{((}\StringTok{"Biomass/Berries"}\NormalTok{), }\StringTok{"summary\_inoc\_overall.csv"}\NormalTok{),}
          \AttributeTok{row.names =} \ConstantTok{FALSE}\NormalTok{)}

\FunctionTok{write.csv}\NormalTok{(summary\_poly\_overall,}
          \FunctionTok{file.path}\NormalTok{((}\StringTok{"Biomass/Berries"}\NormalTok{), }\StringTok{"summary\_poly\_overall.csv"}\NormalTok{),}
          \AttributeTok{row.names =} \ConstantTok{FALSE}\NormalTok{)}
\end{Highlighting}
\end{Shaded}

\subsection{Berry statistical
analysis}\label{berry-statistical-analysis}

I used AI to generate stats. This should be verified for sure!!
Especially to make sure I meet the assumptions to even be using these
stats.

\begin{Shaded}
\begin{Highlighting}[]
\CommentTok{\# These statistics were Chat GPT and should be verified}

\CommentTok{\# P{-}VALUES: Simple comparisons within each Water {-}{-}{-}{-}}
\CommentTok{\# Welch t{-}tests are robust to unequal variances (default in t.test)}

\CommentTok{\# a) Inoculation effect within each Water (folding Polyculture)}
\NormalTok{ttest\_inoc\_within\_water }\OtherTok{\textless{}{-}}\NormalTok{ berries }\SpecialCharTok{\%\textgreater{}\%}
  \FunctionTok{group\_by}\NormalTok{(Water) }\SpecialCharTok{\%\textgreater{}\%}
  \FunctionTok{do}\NormalTok{(}\FunctionTok{tidy}\NormalTok{(}\FunctionTok{t.test}\NormalTok{(Biomass\_g\_pp }\SpecialCharTok{\textasciitilde{}}\NormalTok{ Inoculation, }\AttributeTok{data =}\NormalTok{ .))) }\SpecialCharTok{\%\textgreater{}\%}
  \FunctionTok{ungroup}\NormalTok{() }\SpecialCharTok{\%\textgreater{}\%}
  \FunctionTok{select}\NormalTok{(Water, estimate1, estimate2, statistic, p.value, conf.low, conf.high, method)}

\CommentTok{\# b) Polyculture effect within each Water (folding Inoculation)}
\NormalTok{ttest\_poly\_within\_water }\OtherTok{\textless{}{-}}\NormalTok{ berries }\SpecialCharTok{\%\textgreater{}\%}
  \FunctionTok{group\_by}\NormalTok{(Water) }\SpecialCharTok{\%\textgreater{}\%}
  \FunctionTok{do}\NormalTok{(}\FunctionTok{tidy}\NormalTok{(}\FunctionTok{t.test}\NormalTok{(Biomass\_g\_pp }\SpecialCharTok{\textasciitilde{}}\NormalTok{ Polyculture, }\AttributeTok{data =}\NormalTok{ .))) }\SpecialCharTok{\%\textgreater{}\%}
  \FunctionTok{ungroup}\NormalTok{() }\SpecialCharTok{\%\textgreater{}\%}
  \FunctionTok{select}\NormalTok{(Water, estimate1, estimate2, statistic, p.value, conf.low, conf.high, method)}

\CommentTok{\# {-}{-}{-}{-} 3{-}way factorial model (lets you test interactions) {-}{-}{-}{-}}
\CommentTok{\# If your data are fairly normal{-}ish, this is a strong default.}
\NormalTok{fit\_aov }\OtherTok{\textless{}{-}} \FunctionTok{aov}\NormalTok{(Biomass\_g\_pp }\SpecialCharTok{\textasciitilde{}}\NormalTok{ Inoculation }\SpecialCharTok{*}\NormalTok{ Polyculture }\SpecialCharTok{*}\NormalTok{ Water, }\AttributeTok{data =}\NormalTok{ berries)}
\NormalTok{aov\_table }\OtherTok{\textless{}{-}}\NormalTok{ broom}\SpecialCharTok{::}\FunctionTok{tidy}\NormalTok{(fit\_aov)}

\CommentTok{\# {-}{-}{-}{-} Optional: Nonparametric alternatives (in case assumptions are shaky) {-}{-}{-}{-}}
\CommentTok{\# Wilcoxon rank{-}sum within each Water for Inoculation / Polyculture}
\NormalTok{wilcox\_inoc\_within\_water }\OtherTok{\textless{}{-}}\NormalTok{ berries }\SpecialCharTok{\%\textgreater{}\%}
  \FunctionTok{group\_by}\NormalTok{(Water) }\SpecialCharTok{\%\textgreater{}\%}
  \FunctionTok{do}\NormalTok{(}\FunctionTok{tidy}\NormalTok{(}\FunctionTok{wilcox.test}\NormalTok{(Biomass\_g\_pp }\SpecialCharTok{\textasciitilde{}}\NormalTok{ Inoculation, }\AttributeTok{data =}\NormalTok{ ., }\AttributeTok{exact =} \ConstantTok{FALSE}\NormalTok{))) }\SpecialCharTok{\%\textgreater{}\%}
  \FunctionTok{ungroup}\NormalTok{() }\SpecialCharTok{\%\textgreater{}\%}
  \FunctionTok{select}\NormalTok{(Water, statistic, p.value, method)}

\NormalTok{wilcox\_poly\_within\_water }\OtherTok{\textless{}{-}}\NormalTok{ berries }\SpecialCharTok{\%\textgreater{}\%}
  \FunctionTok{group\_by}\NormalTok{(Water) }\SpecialCharTok{\%\textgreater{}\%}
  \FunctionTok{do}\NormalTok{(}\FunctionTok{tidy}\NormalTok{(}\FunctionTok{wilcox.test}\NormalTok{(Biomass\_g\_pp }\SpecialCharTok{\textasciitilde{}}\NormalTok{ Polyculture, }\AttributeTok{data =}\NormalTok{ ., }\AttributeTok{exact =} \ConstantTok{FALSE}\NormalTok{))) }\SpecialCharTok{\%\textgreater{}\%}
  \FunctionTok{ungroup}\NormalTok{() }\SpecialCharTok{\%\textgreater{}\%}
  \FunctionTok{select}\NormalTok{(Water, statistic, p.value, method)}

\CommentTok{\# Print these}

\NormalTok{ttest\_inoc\_within\_water}
\end{Highlighting}
\end{Shaded}

\begin{verbatim}
## # A tibble: 3 x 8
##   Water    estimate1 estimate2 statistic p.value conf.low conf.high method      
##   <fct>        <dbl>     <dbl>     <dbl>   <dbl>    <dbl>     <dbl> <chr>       
## 1 Control      1.00      1.10     -0.492   0.629   -0.535     0.332 Welch Two S~
## 2 Moderate     1.00      1.11     -0.688   0.501   -0.448     0.229 Welch Two S~
## 3 Severe       0.542     0.412     1.01    0.325   -0.141     0.399 Welch Two S~
\end{verbatim}

\begin{Shaded}
\begin{Highlighting}[]
\NormalTok{ttest\_poly\_within\_water}
\end{Highlighting}
\end{Shaded}

\begin{verbatim}
## # A tibble: 3 x 8
##   Water    estimate1 estimate2 statistic  p.value conf.low conf.high method     
##   <fct>        <dbl>     <dbl>     <dbl>    <dbl>    <dbl>     <dbl> <chr>      
## 1 Control      0.731     1.37      -4.50 0.000468   -0.945    -0.336 Welch Two ~
## 2 Moderate     0.805     1.31      -4.66 0.000377   -0.740    -0.273 Welch Two ~
## 3 Severe       0.268     0.685     -4.79 0.000186   -0.601    -0.233 Welch Two ~
\end{verbatim}

\begin{Shaded}
\begin{Highlighting}[]
\NormalTok{aov\_table}
\end{Highlighting}
\end{Shaded}

\begin{verbatim}
## # A tibble: 8 x 6
##   term                             df    sumsq   meansq statistic   p.value
##   <chr>                         <dbl>    <dbl>    <dbl>     <dbl>     <dbl>
## 1 Inoculation                       1 0.0111   0.0111     0.173    6.79e- 1
## 2 Polyculture                       1 4.08     4.08      63.4      2.51e-10
## 3 Water                             2 4.46     2.23      34.7      4.84e-10
## 4 Inoculation:Polyculture           1 0.000252 0.000252   0.00392  9.50e- 1
## 5 Inoculation:Water                 2 0.184    0.0921     1.43     2.49e- 1
## 6 Polyculture:Water                 2 0.127    0.0634     0.985    3.81e- 1
## 7 Inoculation:Polyculture:Water     2 0.286    0.143      2.22     1.19e- 1
## 8 Residuals                        48 3.09     0.0643    NA       NA
\end{verbatim}

\begin{Shaded}
\begin{Highlighting}[]
\CommentTok{\# Saving as CSV}

\FunctionTok{write.csv}\NormalTok{(ttest\_inoc\_within\_water,}
          \FunctionTok{file.path}\NormalTok{((}\StringTok{"Biomass/Berries/Stats"}\NormalTok{), }\StringTok{"ttest\_inoc\_within\_water.csv"}\NormalTok{),}
          \AttributeTok{row.names =} \ConstantTok{FALSE}\NormalTok{)}

\FunctionTok{write.csv}\NormalTok{(ttest\_poly\_within\_water,}
          \FunctionTok{file.path}\NormalTok{((}\StringTok{"Biomass/Berries/Stats"}\NormalTok{), }\StringTok{"ttest\_poly\_within\_water.csv"}\NormalTok{),}
          \AttributeTok{row.names =} \ConstantTok{FALSE}\NormalTok{)}

\FunctionTok{write.csv}\NormalTok{(aov\_table,}
          \FunctionTok{file.path}\NormalTok{((}\StringTok{"Biomass/Berries/Stats"}\NormalTok{), }\StringTok{"anova\_3way\_table.csv"}\NormalTok{),}
          \AttributeTok{row.names =} \ConstantTok{FALSE}\NormalTok{)}

\FunctionTok{write.csv}\NormalTok{(wilcox\_inoc\_within\_water,}
          \FunctionTok{file.path}\NormalTok{((}\StringTok{"Biomass/Berries/Stats"}\NormalTok{), }\StringTok{"wilcox\_inoc\_within\_water.csv"}\NormalTok{),}
          \AttributeTok{row.names =} \ConstantTok{FALSE}\NormalTok{)}

\FunctionTok{write.csv}\NormalTok{(wilcox\_poly\_within\_water,}
          \FunctionTok{file.path}\NormalTok{((}\StringTok{"Biomass/Berries/Stats"}\NormalTok{), }\StringTok{"wilcox\_poly\_within\_water.csv"}\NormalTok{),}
          \AttributeTok{row.names =} \ConstantTok{FALSE}\NormalTok{)}
\end{Highlighting}
\end{Shaded}


\end{document}
